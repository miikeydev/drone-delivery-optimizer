\documentclass[11pt,a4paper]{article}
\usepackage[T1]{fontenc}
\usepackage{lmodern}
\usepackage{amsmath,amssymb}
\usepackage{geometry}
\usepackage{booktabs}
\usepackage{graphicx}
\usepackage{hyperref}
\geometry{margin=2.5cm}

% Added definitions for missing commands
\newcommand{\E}{\mathbb{E}}
\newcommand{\R}{\mathbb{R}}

\title{\texorpdfstring{Optimisation de tournées de drones à l'aide d'un GNN--PPO\\\small Projet Drone Delivery Optimizer}{Optimisation de tournées de drones à l'aide d'un GNN-PPO - Projet Drone Delivery Optimizer}}
\author{\texorpdfstring{\small Mahouna \_\_ \quad (RTX~4050 / Intel i7)}{Mahouna __ (RTX 4050 / Intel i7)}}
\date{\today}

\begin{document}
\maketitle
\tableofcontents
\newpage

%-------------------------------------------------------------
\section{Introduction}
Ce rapport présente la modélisation mathématique et l'implémentation d'une méthode hybride pour l'optimisation de tournées de drones : 
\begin{enumerate}
  \item un \emph{Algorithme Génétique} (GA) classique corrigé pour garantir les points de départ et d'arrivée,
  \item une approche \textbf{PPO} (Proximal Policy Optimization) alimentée par un \textbf{GNN} pour l'apprentissage par renforcement.
\end{enumerate}

%-------------------------------------------------------------
\section{Modèle de réseau de livraison}
\subsection{Génération des nœuds}
On génère \(N\) nœuds répartis par Poisson–disc à l'intérieur du polygone \emph{France}, issus du fichier \texttt{metropole-version-simplifiee.geojson}.

\subsection{Graphe orienté des \texorpdfstring{\(k\)}{k}-plus-proches-voisins}
Chaque nœud \(u\) est connecté vers ses \(k=10\) plus proches voisins \emph{sortants}, formant un graphe orienté 
\[
  G_t = (\mathcal{V}_t,\mathcal{E}_t),\quad \mathcal{E}_t\subset \{\,u\to v\,\}.
\]
Chaque arête orientée \(e=(u\!\to\!v)\) porte un coût 
\[
  c_{uv}
  = \underbrace{d_{uv}}_{\text{distance}}
  + \underbrace{w_{uv}}_{\text{effet du vent}}
  + \underbrace{\eta_{uv}}_{\text{bruit aléatoire}}
  \;\in[0,600],
\]
où en général \(c_{uv}\neq c_{vu}\).

\subsection{Attributs nodaux \texorpdfstring{\(X\)}{X}}
Pour chaque sommet \(v\), on définit
\[
  x_v = \bigl[
    \underbrace{\text{one-hot}(\text{type})}_{\in\{\,\text{hub},\text{pickup},\text{delivery},\text{charging}\}\, (4)},
    \;
    \underbrace{\text{stock/demande}}_{\in\R\, (1)},
    \;
    \underbrace{\lambda_v,\phi_v}_{\text{latitude/longitude}\,(2)},
    \;
    \underbrace{\cos\omega_v,\sin\omega_v}_{\text{direction du vent}\,(2)}
  \bigr]\;\in\R^{9}.
\]
Le terme « stock/demande » reste nul ou constant ici car on part d'une hypothèse de demande unitaire illimitée.

%-------------------------------------------------------------
\section{Fonction de coût généralisée}
\subsection{Consommation de batterie par arête}
Pour une arête \(e_i\) de coût \(c_i\), la batterie consommée est
\[
  \Delta b_i
  = \frac{c_i}{k}\,\bigl(1 + \alpha\,(p_i - 1)\bigr),
  \quad
  \underbrace{k=10.8}_{\text{normalisation}},\;
  \underbrace{\alpha=0.2}_{\substack{\text{facteur de surcharge}\\\text{(colis multiple)}}},\;
  \underbrace{p_i}_{\text{nombre de colis embarqués}}.
\]
Plus \(p_i\) est grand, plus la consommation augmente.

\subsection{Objectif global d'une tournée \texorpdfstring{\(E\)}{E}}
Pour une séquence d'arêtes \(E=(e_1,\dots,e_T)\), on définit
\[
  J(E)
  =
  \sum_{e_i\in E} c_i
  \;+\;
  \lambda
  \sum_{S}
    \Bigl[\max\bigl(0,\sum_{e_i\in S}\Delta b_i - B_{\max}\bigr)\Bigr]^2
  \;+\;
  \mu\,\#\{\text{recharges}\},
\]
où
\begin{itemize}
  \item \(B_{\max}=100\) est la capacité maximale de batterie,
  \item \(\lambda\gg1\) pénalise fortement tout dépassement de batterie,
  \item \(\mu\ll\lambda\) pénalise légèrement chaque recharge,
  \item \(S\) parcourt chaque segment consécutif entre deux recharges.
\end{itemize}

%-------------------------------------------------------------
\section{Contraintes de début et fin de tournée}

\subsection{1. Algorithme génétique (GA)}
\label{subsec:GA_constraints}

\paragraph{But} Générer des chromosomes (tours) garantissant la séquence :
\[
  H_{\text{start}} \;\to\; D \;\to\; L \;\to\; H_{\text{end}},
\]
avec \(H_{\cdot}\) hubs et \(D,L\) points pickup/delivery.

\paragraph{Initialisation} Pour chaque individu :
\begin{enumerate}
  \item Choix aléatoire d'un hub \(H_{\text{start}}\).
  \item Construction de \(\mathrm{shortest\_path}(H_{\text{start}}\to D)\).
  \item Ajout de \(\mathrm{shortest\_path}(D\to L)\).
  \item Ajout de \(\mathrm{shortest\_path}(L\to H_{\text{end}})\) avec \(H_{\text{end}}\) choisi aléatoirement.
\end{enumerate}
Le chromosome est la concaténation, en omettant les doublons consécutifs.

\paragraph{Réparation (repair)} Après crossover/mutation :
\begin{itemize}
  \item On repère les indices de \(D\) et \(L\).
  \item On sectionne pour forcer \( \dots\to D\to L\to\dots\).
  \item Si la fin n'est pas un hub, on y greffe \(\mathrm{shortest\_path}(\text{dernier}\to H_{\text{rand}})\).
\end{itemize}

\paragraph{Mutation spécifique}  
Échanger deux sous-chemins internes tout en maintenant la séquence \(H\to D\to L\to H\).

%-------------------------------------------------------------
\subsection{2. PPO + GNN (RL)}
\label{subsec:RL_constraints}

\paragraph{Intégration dans l'environnement}  
Au sein de l'environnement de RL :
\begin{itemize}
  \item \emph{État initial virtuel} : l’agent reçoit un état “départ” non relié.
  \item \emph{Pas 0} : action spéciale 
    \[
      a_0 = \bigl(\text{téléportation vers un hub }H_i\bigr),
      \quad \text{coût}=c_{\text{pickup}\to H_i}.
    \]
    Ce hub \(H_i\) est tiré aléatoirement parmi ceux \(\le d_{\max}\) du point de pickup.
  \item \emph{Fin de tournée} : après desserte de \(D\) et \(L\), l’agent choisit
    \[
      a_T = \bigl(\text{téléportation vers hub }H_j\bigr),
      \quad \text{coût}=c_{v_T\to H_j}.
    \]
  \item \emph{Termination} : l’épisode se termine dès que tous les flags \(\{\text{picked},\text{delivered}\}\) sont à 1 et qu’un hub est rejoint.
\end{itemize}

\paragraph{Récompense instantanée mise à jour}
Pour tout pas \(t\),
\[
  r_t 
  = -\,c_t
    - \lambda\,\bigl[\max(0,b_t - \Delta b_t)\bigr]^2
    - \mu\,\mathbb1_{\{\text{recharge}\}}
    - \kappa\,\mathbb1_{\{\text{téléportation non valide}\}},
\]
où \(\kappa\gg1\) pénalise tout téléport illégal hors hub.

%-------------------------------------------------------------
\section{Formulation RL et Architecture du GNN}
\subsection{État \texorpdfstring{\(\displaystyle s_t=(A_t,X_t,b_t,p_t,v_t)\)}{s_t=(A_t,X_t,b_t,p_t,v_t)}}

\subsection{Action \texorpdfstring{\(\displaystyle a_t\in\{1,\dots,K\}\)}{a_t in {1,...,K}}}  
Choix d’une arête orientée sortante ou d’une action spéciale de téléportation.

\subsection{GNN – Message Passing}
\[
  H^{(\ell+1)} 
  = \sigma\bigl(W_1^{(\ell)}H^{(\ell)} + W_2^{(\ell)}A_t\,H^{(\ell)}\bigr),
  \quad 
  H^{(0)}=X_t.
\]

\subsection{Readout global et tête acteur–critique}
\[
  g_t = \frac1{|\mathcal V_t|}\sum_v H_v^{(L)},
\quad
\pi_\theta(a_t\mid s_t)=\mathrm{Softmax}(W_\pi g_t + U_\pi[b_t,p_t]),
\quad
V_\psi(s_t)=w_v^\top g_t + u_v^\top[b_t,p_t].
\]

%-------------------------------------------------------------
\section{Algorithme PPO}
\subsection{Avantage (GAE-\texorpdfstring{\(\lambda\)}{\lambda})}
\[
  \hat A_t 
  = \sum_{k\ge0}(\gamma\lambda)^k\bigl(r_{t+k}+\gamma V_\psi(s_{t+k+1})-V_\psi(s_{t+k})\bigr).
\]

\subsection{Perte « clipped »}
\[
  \mathcal L
  = \E_t\Bigl[\min\bigl(r_t\hat A_t,\;\mathrm{clip}(r_t,1\pm\varepsilon)\hat A_t\bigr)\Bigr]
    +c_1\|R_t-V_\psi\|^2 - c_2\,\mathcal H[\pi_\theta].
\]

%-------------------------------------------------------------
\section{Protocole d'entraînement}
\begin{itemize}
  \item \(N=8\) environnements parallèles.
  \item \(\gamma=0.99,\ \lambda_{\text{GAE}}=0.95,\ \varepsilon=0.2\).
  \item GNN : 2 couches, dimension 128, ReLU, dropout 0.1.
  \item Adam, LR \(3\times10^{-4}\), FP16 sur RTX~4050.
\end{itemize}

%-------------------------------------------------------------
\section{Conclusion}
L’intégration explicite des contraintes de téléportation via un hub garantit la validité des tournées tant dans GA (par génération/réparation) que dans PPO+GNN (par encapsulation dans l’environnement). Le GNN–PPO exploite les coûts orientés \(c_{uv}\) et converge plus rapidement qu’un GA classique tout en respectant les contraintes de batterie grâce aux pénalités \(\lambda\) et \(\mu\).

\end{document}
