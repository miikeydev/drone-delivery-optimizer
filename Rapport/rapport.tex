\documentclass[11pt,a4paper]{article}
\usepackage[T1]{fontenc}
\usepackage{lmodern}
\usepackage{amsmath,amssymb}
\usepackage{geometry}
\usepackage{booktabs}
\usepackage{graphicx}
\usepackage{hyperref}
\usepackage{listings}
\geometry{margin=2.5cm}

% Added definitions for missing commands
\newcommand{\E}{\mathbb{E}}
\newcommand{\R}{\mathbb{R}}

\title{\texorpdfstring{Optimisation de tournées de drones à l'aide d'un GNN--PPO\\\small Projet Drone Delivery Optimizer}{Optimisation de tournées de drones à l'aide d'un GNN-PPO - Projet Drone Delivery Optimizer}}
\author{\texorpdfstring{\small Mahouna \_\_ \quad (RTX~4050 / Intel i7)}{Mahouna __ (RTX 4050 / Intel i7)}}
\date{\today}

\begin{document}
\maketitle
\tableofcontents
\newpage

%-------------------------------------------------------------
\section{Introduction}
Ce rapport présente la modélisation mathématique et l'implémentation d'une méthode hybride pour l'optimisation de tournées de drones : 
\begin{enumerate}
  \item un \emph{Algorithme Génétique} (GA) classique corrigé pour garantir les points de départ et d'arrivée,
  \item une approche \textbf{PPO} (Proximal Policy Optimization) alimentée par un \textbf{GNN} pour l'apprentissage par renforcement.
\end{enumerate}
%-------------------------------------------------------------
\section*{Annexe A \\ Comprendre l'agent GNN--PPO « en une page »}

\textbf{1. Le décor :}  
\begin{itemize}\setlength\itemsep{2pt}
\item Tous les points (hubs, stations de recharge, pickup, delivery) sont placés sur la carte ;
\item Chaque point est relié à ses $k=10$ voisins les plus proches ; chaque flèche possède un coût (distance + vent + bruit) ;
\item Ce réseau orienté est le terrain de jeu : le drone ne peut avancer qu'en suivant les flèches.
\end{itemize}

\textbf{2. La mission d'un épisode :}
\begin{enumerate}\setlength\itemsep{2pt}
\item Se téléporter dans \emph{un} hub de départ ;
\item Rejoindre le pickup et charger ;
\item Livrer au point delivery ;
\item Finir dans \emph{n'importe quel} hub ;
\item Respecter batterie, capacité colis et recharges possibles.
\end{enumerate}

\textbf{3. Prise de décision :}
\begin{description}
\item[a)] Le drone observe le graphe complet, sa position, sa batterie ;
\item[b)] Un GNN condense cela en un vecteur-résumé ;
\item[c)] Un réseau « acteur » en déduit les probabilités de chaque flèche sortante ;
\item[d)] Un réseau « critique » estime la valeur de la situation.
\end{description}


\textbf{4. Apprentissage (PPO) :}  
Des dizaines de drones virtuels explorent en parallèle ; leurs trajectoires alimentent PPO qui met à jour les poids GNN + acteur/critique toutes les quelques itérations, en favorisant les tournées les moins coûteuses et conformes aux contraintes.

\textbf{5. Pourquoi ça marche :}  
Le GNN capture la structure du réseau (vent, distances, hubs), tandis que le RL gère la séquence pickup → delivery → hub sous contraintes. Ensemble, ils découvrent des itinéraires faisables et économes, là où un simple plus-court-chemin échouerait.
%-------------------------------------------------------------

%-------------------------------------------------------------
\section{Modèle de réseau de livraison}

\subsection{Graphe orienté des \texorpdfstring{\(k\)}{k}-plus-proches-voisins}
Chaque nœud \(u\) est connecté vers ses \(k=10\) plus proches voisins \emph{sortants}, formant un graphe orienté 
\[
  G_t = (\mathcal{V}_t,\mathcal{E}_t),\quad \mathcal{E}_t\subset \{\,u\to v\,\}.
\]
Chaque arête orientée \(e=(u\!\to\!v)\) porte un coût 
\(
  c_{uv}
  = d_{uv} \cdot \Bigl(
  \underbrace{1}_{\text{base}}
  \underbrace{- \alpha \cdot w_{uv}}_{\text{effet du vent}}
  \underbrace{+ \beta \cdot \eta_{uv}}_{\text{bruit aléatoire}}
  \Bigr)
  \;\in[0,600],
\)
où en général \(c_{uv}\neq c_{vu}\).

\subsection*{Implémentation du graphe dans le code}
Le graphe orienté des \(k\)-plus-proches-voisins est construit dans le code JavaScript (\texttt{app.js}) selon les étapes suivantes :
\begin{enumerate}
  \item \textbf{Collecte des nœuds} : tous les points (hubs, charging, delivery, pickup) sont stockés dans un tableau \texttt{allNodes} avec leurs coordonnées et leur type.
  \item \textbf{Recherche des voisins} : pour chaque nœud \(u\), on calcule la distance de Haversine à tous les autres nœuds, puis on trie ces distances pour sélectionner les \(k=10\) plus proches voisins (hors lui-même).
  \item \textbf{Création des arêtes orientées} : pour chaque voisin \(v\) sélectionné, une arête orientée \((u \to v)\) est ajoutée à la liste des arêtes.
  \item \textbf{Calcul du coût d'arête} : chaque arête reçoit un coût $c_{uv}$ qui dépend de la distance, de l'effet du vent (calculé via l'angle entre l'arête et la direction du vent global), et d'un bruit aléatoire. Ceci est réalisé dans la fonction \texttt{annotateEdges}.
  \item \textbf{Affichage} : le graphe est affiché sur la carte avec des couleurs et épaisseurs d'arêtes reflétant le coût, et chaque arête possède un tooltip interactif.
\end{enumerate}
Ce procédé garantit que le graphe utilisé pour l'optimisation correspond exactement à la structure mathématique décrite précédemment.

\subsection{Attributs nodaux \texorpdfstring{\(X\)}{X}}
Pour chaque sommet \(v\), on définit
\[
  x_v = \bigl[
    \underbrace{\mathrm{one\text{-}hot}(\mathrm{type})}_{\in\{\,\text{hub},\text{pickup},\text{delivery},\text{charging}\}\, (4)},
    \;
    \underbrace{\lambda_v,\phi_v}_{\text{latitude/longitude}\,(2)},
    \;
    \underbrace{\text{stock/demande}}_{\in\R\, (1)}
  \bigr]\;\in\R^{7}.
\]
Le terme « stock/demande » reste nul ou constant ici car on part d'une hypothèse de demande unitaire illimitée.

%-------------------------------------------------------------
\section{Fonction de coût généralisée}
\subsection{Effet du vent sur les coûts d'arêtes}
Le vent est considéré comme un paramètre global (même direction partout), et son effet est calculé au niveau de chaque arête orientée \((u\to v)\) via:
\[
  \text{windEffect}_{uv} = \cos(\phi_{uv} - \text{windAngle})
\]
où \(\phi_{uv}\) est l'angle de la direction \(u\to v\) et \(\text{windAngle}\) est la direction globale du vent. Cette valeur est ensuite utilisée dans le calcul du coût de l'arête présenté précédemment.

\subsection{Consommation de batterie par arête}
Pour une arête \(e_i\) de coût \(c_i\), la batterie consommée est
\[
  \Delta b_i
  = \frac{c_i}{k}\,\bigl(1 + \alpha\,(p_i - 1)\bigr),
  \quad
  \underbrace{k=10.8}_{\text{normalisation}},\;
  \underbrace{\alpha=0.2}_{\substack{\text{facteur de surcharge}\\\text{(colis multiple)}}},\;
  \underbrace{p_i}_{\text{nombre de colis embarqués}}.
\]
Plus \(p_i\) est grand, plus la consommation augmente.

\subsection{Objectif global d'une tournée \texorpdfstring{\(E\)}{E}}
Pour une séquence d'arêtes \(E=(e_1,\dots,e_T)\), on définit
\[
  J(E)
  =
  \sum_{e_i\in E} c_i
  \;+\;
  \lambda
  \sum_{S}
    \Bigl[\max\bigl(0,\sum_{e_i\in S}\Delta b_i - B_{\max}\bigr)\Bigr]^2
  \;+\;
  \mu\,\#\{\text{recharges}\},
\]
où
\begin{itemize}
  \item \(B_{\max}=100\) est la capacité maximale de batterie,
  \item \(\lambda\gg1\) pénalise fortement tout dépassement de batterie,
  \item \(\mu\ll\lambda\) pénalise légèrement chaque recharge,
  \item \(S\) parcourt chaque segment consécutif entre deux recharges.
\end{itemize}

%-------------------------------------------------------------
\section{Contraintes de début et fin de tournée}

\subsection{1. Algorithme génétique (GA)}
\label{subsec:GA_constraints}

\paragraph{But} Générer des chromosomes (tours) garantissant la séquence :
\[
  H_{\text{start}} \;\to\; D \;\to\; L \;\to\; H_{\text{end}},
\]
avec \(H_{\cdot}\) hubs et \(D,L\) points pickup/delivery.

\paragraph{Initialisation} Pour chaque individu :
\begin{enumerate}
  \item Choix aléatoire d'un hub \(H_{\text{start}}\).
  \item Construction de \(\mathrm{shortest\_path}(H_{\text{start}}\to D)\).
  \item Ajout de \(\mathrm{shortest\_path}(D\to L)\).
  \item Ajout de \(\mathrm{shortest\_path}(L\to H_{\text{end}})\) avec \(H_{\text{end}}\) choisi aléatoirement.
\end{enumerate}
Le chromosome est la concaténation, en omettant les doublons consécutifs.

\paragraph{Réparation (repair)} Après crossover/mutation :
\begin{itemize}
  \item On repère les indices de \(D\) et \(L\).
  \item On sectionne pour forcer \( \dots\to D\to L\to\dots\).
  \item Si la fin n'est pas un hub, on y greffe \(\mathrm{shortest\_path}(\text{dernier}\to H_{\text{rand}})\).
\end{itemize}

\paragraph{Mutation spécifique}  
Échanger deux sous-chemins internes tout en maintenant la séquence \(H\to D\to L\to H\).

%-------------------------------------------------------------
\subsection{2. PPO + GNN (RL)}


%-------------------------------------------------------------
\subsection{Modélisation RL détaillée}
\label{sec:mdp}

\paragraph{Cadre MDP.}
Le problème d'optimisation est formulé comme un \emph{processus de décision de Markov} 
$\mathcal{M}=(\mathcal{S},\mathcal{A},P,r,\gamma)$ :

\begin{table}[h]
\centering\small
\begin{tabular}{@{}p{3cm}p{11cm}@{}}
\toprule
\textbf{Ensemble} & \textbf{Définition précise} \\
\midrule
États $s_t$ &
$(A,\;X,\;E,\;b_t,\;p_t,\;v_t,\;f_{\text{pick}},\;f_{\text{deliv}})$ avec :
\begin{itemize}\setlength\itemsep{2pt}
  \item $A\!\in\!\{0,1\}^{N\times N}$ : matrice d'adjacence orientée ;
  \item $X\!\in\!\mathbb{R}^{N\times7}$ : features nodales 
        \small(one-hot(type), lat, lon, stock)\normalsize ;
  \item $E\!\in\!\mathbb{R}^{|\mathcal{E}|\times3}$ : features d'arête 
        \small(distance,\,cos(angle\_vent),\,bruit)\normalsize ;
  \item $b_t\!\in\![0,B_{\max}]$ : batterie restante ;
  \item $p_t\!\in\!\{0,\dots,p_{\max}\}$ : charge à bord ;
  \item $v_t$ : index du nœud courant ;
  \item $f_{\text{pick}},f_{\text{deliv}}\!\in\!\{0,1\}$ : flags « pickup fait / delivery fait ».
\end{itemize}\\
\addlinespace
Actions $a_t$ &
\textbf{Pas 0} : choisir un hub $h_i\in\mathcal{H}$ (téléportation).  
\textbf{Pas $t\ge1$} : choisir l'une des arêtes sortantes 
$\mathcal{N}^{\text{out}}(v_t)$ du nœud courant.  
Optionnel : action \texttt{RECHARGE} disponible uniquement si $v_t$ est une station.\\
\addlinespace
Transition &
\begin{enumerate}\setlength\itemsep{2pt}
  \item Si $t=0$ : $v_{t+1}\!\gets\!h_i$ (hub choisi).  
  \item Sinon : $v_{t+1}$ = voisin choisi.  
  \item Mise à jour batterie : $b_{t+1}\!=\!b_t-\Delta b_t$ (cf. §\,\ref{sec:rl_gnn}).  
  \item Recharge automatique si $v_{t+1}\in\mathcal{C}$ (*charging*).  
  \item Flags : \,pickup $\rightarrow$ $f_{\text{pick}}{:=}1$ \;| delivery $\rightarrow$ $f_{\text{deliv}}{:=}1,\;p_{t+1}{:=}0$.
\end{enumerate}\\
\addlinespace
Récompense $r_t$ &
$-\;c_{v_t v_{t+1}}
-\lambda\,[\max(0,\;b_{t+1}{<}0)]^{2}
-\mu\;\mathbb{1}_{\{\text{recharge}\}}$ \\
\addlinespace
Condition d'arrêt &
\(
(f_{\text{pick}}{=}1\;\land\;f_{\text{deliv}}{=}1\;\land\;v_t\in\mathcal{H})\;\lor\;t=T_{\max}
\)\\
\bottomrule
\end{tabular}
\caption{Résumé des composantes du MDP utilisé pour PPO.}
\end{table}

\paragraph{Masque d'action dynamique.}
À chaque pas, la distribution de politique est calculée \emph{uniquement} sur les arêtes
autorisées ; les autres logits sont fixés à $-10^{9}$ avant la softmax.

\paragraph{Réseau de politique.}
La combinaison \emph{Edge-conditioned GraphSAGE + MLP} décrite dans
la sous-section précédente réalise la paire acteur–critique.  
Le vecteur d'entrée final est :
\[
\bigl[g_t,\;b_t,\;p_t,\;f_{\text{pick}},\;f_{\text{deliv}},\;\text{one-hot}(v_t)\bigr].
\]

\paragraph{Pseudo-code environnement (Gym).}
\vspace{-3pt}
\begin{lstlisting}[language=Python,basicstyle=\ttfamily\small,frame=single]
class DroneDeliveryEnv(gym.Env):
    def reset(self):
        self.v = 0              # noeud virtuel
        self.batt = B_max
        self.colis, self.picked, self.deliv = p_max, 0, 0
        self.t = 0
        return self._obs()

    def step(self, action):
        if self.t == 0:
            self.v = hubs[action]
        else:
            self.v = out_neighbors[self.v][action]
        cost = edge_cost(prev, self.v)
        self.batt -= batt_cost(cost)
        if self.v == pickup and not self.picked:
            self.picked = 1
        if self.v == delivery and self.picked and not self.deliv:
            self.deliv, self.colis = 1, 0
        if node_type[self.v] == CHARGING:
            self.batt = B_max
        done = (self.picked and self.deliv and node_type[self.v]==HUB) \
               or self.t >= T_max
        reward = -cost - lam * (self.batt < 0)**2 - mu * (node_type[self.v]==CHARGING)
        self.t += 1
        return self._obs(), reward, done, False, {}
\end{lstlisting}
\vspace{-6pt}

\paragraph{Pipeline d'entraînement.}
Le modèle \texttt{PolicyGNN} est entraîné via PPO
avec \(|\mathcal{E}|\) environnements parallèles (cf. listing \ref{lst:ppo}).

\begin{lstlisting}[language=Python,basicstyle=\ttfamily\small,frame=single,caption={Extraction minimale du script d'entraînement},label={lst:ppo}]
envs = SubprocVecEnv([make_env(i) for i in range(num_envs)])
model = PPO(PolicyGNN, envs,
            n_steps=2048, batch_size=256,
            learning_rate=3e-4, gamma=0.99,
            gae_lambda=0.95, clip_range=0.2)
model.learn(total_timesteps=5_000_000)
\end{lstlisting}
%-------------------------------------------------------------

\label{sec:rl_gnn}

\paragraph{Intégration des contraintes (mise à jour)}  
\begin{itemize}
  \item \emph{Pas 0 (téléportation)} : comme avant,  
    \[
      a_0 = (\text{téléportation vers hub }H_i),\quad H_i\sim\{H\mid d(\text{pickup},H)\le d_{\max}\}.
    \]
  \item \emph{Étapes intermédiaires} (\(1\le t \le T\)) : on suit une arête \((v_t\to v_{t+1})\).
  \item \emph{Termination} : l'épisode se termine dès que  
    \[
      \text{\texttt{picked}}=\text{\texttt{delivered}}=1
      \quad\text{et}\quad
      v_t\in\mathcal{H}
      \quad\text{ou}\quad
      t=T_{\max}.
    \]
\end{itemize}

L'agent doit naturellement rejoindre un hub après avoir desservi tous les points. Cela rend la terminaison plus réaliste et évite d'introduire une action artificielle de retour. Un critère temporel optionnel (\(T_{\max}\)) permet aussi de forcer la fin de l'épisode si besoin.

\subsection{Encodage de l'état \(s_t\)}
L'état est un tuple 
\[
  s_t = \bigl(A_t,\,X_t,\,E_t,\,b_t,\,p_t,\,v_t\bigr),
\]
\begin{itemize}
  \item \(A_t\in\{0,1\}^{N\times N}\) : matrice d'adjacence orientée du graphe,
  \item \(X_t\in\R^{N\times d_0}\) : matrice des \emph{features nodales},   
    \[
      x_v = [\,\text{one-hot(type)},\ \lambda_v,\ \phi_v,\ \text{stock/demande}\,]
    \]
    où :
    \begin{itemize}
      \item \(\text{one-hot(type)}\) : vecteur indicateur (hub, pickup, delivery, charging), dimension 4,
      \item \(\lambda_v\) : latitude,
      \item \(\phi_v\) : longitude,
      \item \(\text{stock/demande}\) : valeur de stock ou demande (ici constante).
    \end{itemize}
  \item \(E_t\in\R^{|\mathcal{E}|\times f_e}\) : matrice des \emph{features d'arête},  
    pour chaque \(e=(u\to v)\) on encode  
    \[
      e_{uv}=\bigl[d_{uv},\,\cos(\phi_{uv}-\omega),\,\eta_{uv}\bigr],
    \]
  \item \(b_t\in\R\) : batterie restante,
  \item \(p_t\in\mathbb{N}\) : nombre de colis à bord,
  \item \(v_t\in\{1,\dots,N\}\) : index du nœud courant.
\end{itemize}

\subsection{Architecture du GNN}
On utilise un schéma \emph{Message Passing} enrichi par les features d'arête (Edge-conditioned GraphSAGE) :
\[
  H^{(\ell+1)} 
  = \sigma\Bigl(
    W_1^{(\ell)}H^{(\ell)}
    \;+\;
    W_2^{(\ell)}\,\sum_{u\to v}\!\phi_\theta\bigl(e_{uv}\bigr)\,H_u^{(\ell)}
  \Bigr),
  \quad
  H^{(0)} = X_t,
\]
avec \(\phi_\theta\) un réseau MLP qui transforme la feature d'arête \(e_{uv}\) en un poids d'agrégation.  

\subsection{Readout et têtes acteur–critique}
Après \(L\) couches GNN, on agrège par 
\[
  g_t = \frac{1}{N}\sum_{v=1}^N H_v^{(L)} \in \R^d.
\]
\paragraph{Politique (acteur)}  
\[
  \pi_\theta(a_t\mid s_t)
  = \mathrm{Softmax}\Bigl(W_\pi\,g_t\;+\;U_\pi\,[b_t,p_t]\Bigr).
\]
\paragraph{Critique}  
\[
  V_\psi(s_t)
  = w_v^\top g_t \;+\; u_v^\top [b_t,p_t].
\]

\subsection{Récompense instantanée}
Pour tout pas \(t\),
\[
  r_t
  = -c_t
  - \lambda\,\Bigl[\max\bigl(0,\,b_t - \Delta b_t\bigr)\Bigr]^2
  - \mu\,\mathbb{1}_{\{\text{recharge}\}}
  - \kappa\,\mathbb{1}_{\{\text{téléport. hors hub}\}},
\]
avec \(\kappa\gg1\) pour interdire toute téléportation hors hub.

%-------------------------------------------------------------
\section{Algorithme PPO}
\subsection{Avantage (GAE-\(\lambda\))}
\[
  \hat{A}_t 
  = \sum_{k\ge0}(\gamma\lambda)^k
    \Bigl(r_{t+k} 
    + \gamma V_\psi(s_{t+k+1})
    - V_\psi(s_{t+k})\Bigr).
\]
\subsection{Perte « clipped »}
\[
  \mathcal{L}
  = \E_t\Bigl[
      \min\bigl(r_t(\theta)\,\hat{A}_t,\;
                 \mathrm{clip}(r_t(\theta),1\pm\varepsilon)\,\hat{A}_t
           \bigr)
    \Bigr]
  + c_1\,\|R_t - V_\psi\|^2
  - c_2\,\mathcal{H}\bigl[\pi_\theta\bigr].
\]

\subsection{Pipeline d'entraînement}
\begin{enumerate}
  \item \textbf{Collecte de trajectoires} sur \(N\) environnements parallèles.
  \item \textbf{Calcul des avantages} \(\{\hat{A}_t\}\) via GAE-\(\lambda\).
  \item \textbf{Optimisation conjointe} des paramètres \(\theta,\psi\) sur plusieurs epochs :
    \[
      \theta\leftarrow\theta-\eta\,\nabla_\theta\mathcal{L},\quad
      \psi\leftarrow\psi-\eta\,\nabla_\psi\bigl\|R_t - V_\psi\bigr\|^2.
    \]
  \item Boucler jusqu'à convergence.
\end{enumerate}

\paragraph{Hyperparamètres typiques}
\begin{itemize}
  \item \(\gamma=0.99,\ \lambda_{\text{GAE}}=0.95,\ \varepsilon=0.2\).
  \item GNN : \(L=2\) couches, \(d=128\), ReLU, dropout \(0.1\).
  \item Optimiseur : Adam, LR \(3\times10^{-4}\), FP16 sur GPU.
\end{itemize}

%-------------------------------------------------------------
\section{Conclusion}
L'intégration explicite des contraintes de téléportation via un hub garantit la validité des tournées tant dans GA (par génération/réparation) que dans PPO+GNN (par encapsulation dans l'environnement). Le GNN–PPO exploite les coûts orientés \(c_{uv}\) et converge plus rapidement qu'un GA classique tout en respectant les contraintes de batterie grâce aux pénalités \(\lambda\) et \(\mu\).

\end{document}



